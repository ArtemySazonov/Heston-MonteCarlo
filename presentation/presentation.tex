\documentclass[aspectratio=169]{beamer}
\usetheme{vega}

\addbibresource{assets/pricing_lib.bib}

\DeclareMathOperator*{\plim}{\ensuremath{\operatorname{\P-lim}}}

\newcommand{\cA}{\mathcal{A}}
\newcommand{\cB}{\mathcal{B}}
\newcommand{\cC}{\mathcal{C}}
\newcommand{\cN}{\mathcal{N}}

\subtitle{Student Research Group 'Stochastic Volatility Models', Project 'Heston--2'}
\title{Exotics Pricing via the Simulation of the Heston Model}
\author{Artemy Sazonov, Danil Legenky, Kirill Korban}
\institute{Lomonosov Moscow State Univesity, Faculty of Mechanics and Mathematics}
\date{February 25, 2023}

\begin{document}
    \maketitle

    \begin{frame}{Heston Model Definition}
        Assume that the spot asset at time $t$ follows the diffusion
        \begin{align}
            dS(t) & = \mu S(t)dt + \sqrt{v(t)} S(t) dZ_1(t), \label{Heston:price}\\
            dv(t) & = \left(\delta^2 - 2\beta v(t)\right) dt + \sigma\sqrt{v(t)} dZ_2(t), \label{Heston:variance}
        \end{align}
        where $Z_1$, $Z_2$ are the correlated Wiener processes with $dZ_1dZ_2 = \rho dt$.
    \end{frame}

    \section{Introduction to Monte-Carlo Methods}
        \subsection{Monte Carlo Simulation}
    \begin{frame}{Monte Carlo Simulation}{Statistical Estimation}
        \begin{lemma}
            Let $X_1, X_2, \dots, X_n$ be a series of independent and identically distributed random variables, and $h: \mathbb{R} \to \mathbb{R}$ be a borel function. Then $h(X_1), h(X_2), \dots, h(X_n)$ is a series of independent and identically distributed random variables.
        \end{lemma}
        Thus, we could write an unbiased consistent estimator of $\E \left[h(X)\right]$ as follows:
        \begin{equation}
            \widehat{\E \left[h(X)\right]} = \frac{1}{n} \sum_{i=1}^n h(X_i).
        \end{equation}
    \end{frame}

    \begin{frame}{Monte Carlo Simulation}{Local Truncation Error}
        \begin{definition}
            Monte Carlo simulation is a set of techniques that use pseudorandom number generators to solve problems that might be too complicated to be solved analytically. It is based on the central limit theorem.
        \end{definition}
        Asymptotic confidence interval for $\hat{\mu} = \widehat{\E\left[X\right]}$ at the confidence level $\alpha$:
        \begin{equation}
            \mu \in \left(\hat{\mu} - z_{\alpha/2} \sqrt{\frac{\sigma^2}{n}}, \hat{\mu} + z_{\alpha/2} \sqrt{\frac{\sigma^2}{n}}\right).
        \end{equation}
        That means that the estimation error is equal to $2z_{\alpha/2} \sqrt{\frac{\sigma^2}{n}}$.
    \end{frame}

    \begin{frame}{Discretization Schemes for SDEs}{Strong and weak convergence as a global truncation error analogue}
        \begin{definition}
            Let $\hat X^n(t)$ be a mesh approximation of an SDE solution $X(t)$ (we assume that there exists a unique strong solution). 
            Then a scheme is said to have a strong convergence of order $p$ if 
            \begin{equation}
                \E\left[\left|\hat X^n(T) - X(T)\right|\right] \leq Ch^p, \quad n \to \infty.
            \end{equation}
            A scheme is said to have a weak convergence of order $p$ if for any polynomial $f: \R \to \R$ we have
            \begin{equation}
                \left|\E\left[f(\hat X^n(T))\right] - \E\left[f(X(T))\right]\right| \leq Ch^p, \quad n \to \infty.
            \end{equation}
        \end{definition}
    \end{frame}

    \section{Euler Simulation Method}
        \begin{frame}{Euler Scheme for the Heston Model}{Modified Euler-Maruyama Discretization Scheme}
    Suppose we have the Heston model \eqref{Heston:price} -- \eqref{Heston:variance}. Then it could be numerically solved by the following finite difference scheme (for the log-prices $X(t)$):
    \begin{align}
        X_{n+1} & = X_n + (\mu - 0.5 v_n^+)h_n + \sqrt{v_n^+} \sqrt{h_n} Z_{1,n}, \label{Euler:Heston:price:posmod}\\
        v_{n+1} & = v_n + \left(\delta^2 - 2\beta v_n^+\right) h_n + \sigma \sqrt{v_n^+} \sqrt{h_n} Z_{2,n}, \label{Euler:Heston:variance:posmod}
    \end{align}
    and then we take the exponential of the log-prices:
    \begin{equation}
        S_{n} = S_0 e^{X_{n}}.
    \end{equation}
    
    However, the scheme is not accurate, since we ignore the $dZ_idZ_j$ terms in the It\^o-Taylor series approximation.
\end{frame}

    \section{Andersen Simulation Methods}
        \input{part3/part3.tex}

    \section{Computation Examples}
        \subsection{Control Variates}
    \begin{frame}{Control Variates}{Params \#3: $\kappa = 0.5$, $\gamma = 1$, $\rho = -0.9$, $\bar v = 0.04$, $v_0 = 0.04$, \texttt{absolute\_error = 5e-2}}
        \begin{figure}
            \includegraphics[width=\textwidth]{part5/pictures/surf_3.pdf}
        \end{figure}
    \end{frame}

    \begin{frame}{Control Variates}{Params \#3: $\kappa = 0.5$, $\gamma = 1$, $\rho = -0.9$, $\bar v = 0.04$, $v_0 = 0.04$, \texttt{absolute\_error = 5e-2}}
        \begin{figure}
            \includegraphics[width=\textwidth]{part5/pictures/pot_3.pdf}
        \end{figure}
    \end{frame}

    \begin{frame}{Control Variates}{Params \#4: $\kappa = 0.3$, $\gamma = 0.9$, $\rho = -0.5$, $\bar v = 0.04$, $v_0 = 0.04$, \texttt{absolute\_error = 5e-2}}
        \begin{figure}
            \includegraphics[width=\textwidth]{part5/pictures/surf_4.pdf}
        \end{figure}
    \end{frame}

    \begin{frame}{Control Variates}{Params \#4: $\kappa = 0.3$, $\gamma = 0.9$, $\rho = -0.5$, $\bar v = 0.04$, $v_0 = 0.04$, \texttt{absolute\_error = 5e-2}}
        \begin{figure}
            \includegraphics[width=\textwidth]{part5/pictures/pot_4.pdf}
        \end{figure}
    \end{frame}

    \begin{frame}{Control Variates}{Params \#5: $\kappa = 1$, $\gamma = 1$, $\rho = -0.3$, $\bar v = 0.04$, $v_0 = 0.09$, \texttt{absolute\_error = 5e-2}}
        \begin{figure}
            \includegraphics[width=\textwidth]{part5/pictures/surf_5.pdf}
        \end{figure}
    \end{frame}

    \begin{frame}{Control Variates}{Params \#5: $\kappa = 1$, $\gamma = 1$, $\rho = -0.3$, $\bar v = 0.04$, $v_0 = 0.09$, \texttt{absolute\_error = 5e-2}}
        \begin{figure}
            \includegraphics[width=\textwidth]{part5/pictures/pot_5.pdf}
        \end{figure}
    \end{frame}



    \section{Conclusion}
        \begin{frame}{Conclusion}
            \begin{enumerate}
                \item We introduced the three common Heston simulation methods: Euler-Maruyama, Andersen TG and Andersen QE; 
                \item We compared the theoretical vanilla options prices and their Monte-Carlo counterparts;
                \item We measured the performance while pricing the exotics.
            \end{enumerate}
        \end{frame}
        \begin{frame}{To-dos}
            \begin{enumerate}
                \item Implement the Exact (Broadie and Kaya) scheme;
                \item Measure the presiceness of the pricers for the real market data;
                \item Part of the pricing library.
            \end{enumerate}
        \end{frame}

\end{document}
